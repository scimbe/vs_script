\section{VS Definition}
\subsection{Definitionen in der Nutzung}
\begin{frame}
  \frametitle{Definitionen}
  \framesubtitle{Beispiele}
  \begin{itemize}
    \item Eine Sammlung unabhängiger Computer, die den Benutzern als ein kohärentes System erscheinen - Tanenbaum
    \item Mehrere unabängige Computer, die miteinander kommunizieren und kooperieren, um gemeinsam eine Aufgabe zu erfüllen - Garg
    \item Ein Netzwerk aus autonomen Computern, die miteinander kommunizieren und koordiniert zusammenarbeiten, um eine gemeinsame Aufgabe zu erfüllen - Mukherjee
  \end{itemize}
\end{frame}

\subsection{Definitionen plural?}
\begin{frame}
  \frametitle{Definitionen}
  \framesubtitle{Definitionen plural?}
  \begin{itemize}
    \item Wichtig: Nicht gemeinsam genutzter Speicher
    \item Definition muss im Prozess vervollständigt werden
    \item Eigenschaften wichtig für korrekten Lösungsbaum (Labyrinth)
  \end{itemize}
\end{frame}


\subsection{Abgrenzung II}
\begin{frame}
  \frametitle{Abgrenzung Verteiltes und Monolithisches System}
  \framesubtitle{Monolithisches System}
  \begin{itemize}
  \item Vorteile:
    \begin{itemize}
    \item Einfachere Architektur und Implementierung
    \item Geringere Anforderungen an die Netzwerk- und Kommunikationsinfrastruktur
    \end{itemize}
  \item Nachteile:
    \begin{itemize}
    \item Begrenzte Skalierbarkeit und Verfügbarkeit
    \item Anfälliger für Ausfälle
    \item Kann zu Single-Point-of-Failure-Situationen führen
    \end{itemize}
  \end{itemize}
\end{frame}

\begin{frame}
  \frametitle{Abgrenzung Verteiltes und Monolithisches System}
  \framesubtitle{Verteiltes System}

  \begin{itemize}
  \item Vorteile:
    \begin{itemize}
    \item Bessere Skalierbarkeit und Verfügbarkeit
    \item Robuster und widerstandsfähiger gegenüber Ausfällen
    \item Ermöglicht die gemeinsame Nutzung von Ressourcen mehrerer Computer
    \end{itemize}
  \item Nachteile:
    \begin{itemize}
    \item Komplexere Architektur und höhere Komplexität in der Implementierung
    \item Höhere Anforderungen an die Netzwerk- und Kommunikationsinfrastruktur
    \end{itemize}
  \end{itemize}
\end{frame}

\begin{frame}
  \frametitle{Abgrenzung Großrechner}
  \framesubtitle{Verteiltes System}
  \begin{itemize}
  \item Mehrere unabhängige Computer vs. Großrechner: eine Computeranlage
  \item Höhere Skalierbarkeit und Verfügbarkeit vs. Großrechner: auf eine Computeranlage beschränkt
  \item Robuster und widerstandsfähiger gegenüber Ausfällen vs. Großrechner: anfälliger für Ausfälle
  \item Höhere Flexibilität in der Software-Entwicklung vs. Großrechner: oft auf eine bestimmte Architektur und Plattform beschränkt
  \item Großrechner erfordern oft teurere spezialisierte Hardware und Software
  \end{itemize}

\end{frame}


\subsection{Abstraktionsebenen}
\begin{frame}
  \frametitle{Abstraktionsebenen}
  \framesubtitle{Grundlage der Analyse}
  \begin{itemize}
    \item Technologische Ebene 
    \item Anwendungsebene
  \end{itemize}
\end{frame}

\begin{frame}
  \frametitle{Abstraktionsebenen}
  \framesubtitle{AI und ITS}
  \begin{itemize}
    \item ITS: Generisch bildet der Schwerpunkt die Entwicklung von Algorithmen und Protokollen
    \item ITS: Beispiele: Auswahl Kommunikations-Technologien, Architektur Infrastruktur 
    \item AI: Generisch bildet der Schwerpunkt die Anwendung von vorhandenen Technologien 
    \item AI:  Beispiele aus Anwendungsbereich: Cloud, E-Commerce, Datenbanken
  \end{itemize}
\end{frame}

\subsection{Aspekte und Sichten}
\begin{frame}
  \frametitle{Aspekte und Sichten}
  \framesubtitle{Aspekte}
  \begin{itemize}
    \item Skalierbarkeit und Ausfallsicherheit
    \item Datenmanagement
    \item Orchestrierung und Deployment
    \item Sicherheit
  \end{itemize}
\end{frame}

\begin{frame}
  \frametitle{Aspekte und Sichten}
  \framesubtitle{Sichten}
  \begin{itemize}
    \item Architektursicht
    \item Prozesssicht
    \item Datensicht
    \item Sicherheitssicht
    \item Betriebssicht
    \item Entwicklersicht
  \end{itemize}
\end{frame}

\begin{frame}
  \frametitle{Einfluss von Sichten auf Entwicklung}
  \framesubtitle{Beispiel DevOps}
  \begin{itemize}
    \item DevOps ist eine agile Methode
    \item Zusammenarbeit zwischen der Entwicklung (Dev) und dem Betrieb (Ops)
    \item Enge Zusammenarbeit und Kommunikation zwischen den Teams
    \item Neue Werkzeuge wie: Continuous Integration (CI) und Continuous Delivery (CD)
    \item Hohe Einarbeitungskosten
    \item Hohe kulturelle Herausforderungen
  \end{itemize}
\end{frame}
