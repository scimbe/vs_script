
% !TEX options=--shell-escape
\immediate\write18{makeindex \jobname.idx \jobname.glo -t \jobname.glg -s \jobname.ist -o \jobname.gls}
%%%%%%%%%%%%%% Do not touch   %%%%%%%%%%%%%%%%%%%%%%%



%%%%%%%%%%%%%% OALS DOWNLOAD %%%%%%%%%%%%%%%%%%%%%%%
%\documentclass[a4paper]{article}


%%%%%%%%%%% ALS BUCH %%%%%%%%%%%%%%%%%%%%
\documentclass[a4paper,10pt]{book}
\setcounter{chapter}{2}
\renewcommand{\thesection}{\arabic{section}}

\usepackage[margin=3cm]{geometry} %Ändert alle Ränder auf 2cm
% oder Sie können einzelne Ränder so anpassen:
%\usepackage[left=3cm, right=2.5cm, top=3cm, bottom=3cm]{geo\textbf{metry}

% Pakete
\usepackage[utf8]{inputenc}
\usepackage[T1]{fontenc}
\usepackage[german]{babel}
\usepackage{amsmath,amssymb,amsthm}
\usepackage{subcaption} 
\usepackage{listings}
\usepackage{color}
\usepackage{showidx}
\usepackage{imakeidx}
\usepackage[version=4]{mhchem}
\usepackage{listings}
\usepackage{color}
\usepackage{xcolor} 
\usepackage{plantuml}
\usepackage{csquotes}
\usepackage{enumitem} \setitemize{leftmargin=*}   
\usepackage{pgfplots}
\usepackage{amsmath}
\usepackage{tabularx}
\usepackage{microtype}


\usepackage{import}
\usepackage{graphicx}
\usepackage{tcolorbox}
\usepackage{fancyhdr}
\usepackage{booktabs}
\usepackage{lastpage}
\usepackage{float}
\usepackage{caption}
\usepackage{hyperref}

%%%%%%%%%%%%%% New Commands   %%%%%%%%%%%%%%%%%%%%%%%
\newcommand*{\subfilesbibliography}[1]{%
  {\bibliography{#1}}
  {}%
}

\newcommand{\importantvs}[1]{
  \begin{tcolorbox}[colback=yellow!10!white,colframe=blue!50!black]
    #1
  \end{tcolorbox}
}
\newcommand{\examplevs}[1]{
  \begin{tcolorbox}[colback=blue!10!white, colframe=blue!50!black]
    #1
  \end{tcolorbox}
}
\newcommand{\warningvs}[1]{
  \begin{tcolorbox}[colback=red!10!white,colframe=blue!50!black]
    #1
  \end{tcolorbox}
}

\pagestyle{fancy}
\renewcommand{\sectionmark}[1]{\markright{#1}} % Aktualisieren Sie den rechten Markierungsbefehl, um den Abschnittstitel zu speichern
\fancyhf{} % Löscht die aktuellen Kopf- und Fußzeilen
\fancyhead[R]{\rightmark} % Setzen Sie den rechten Kopfzeileninhalt auf den Abschnittstitel
\fancyfoot[C]{\thepage{} von \pageref*{LastPage}} % Setzt die Seitenzahlen in der Fußzeile
\fancyfoot[R]{\copyright\ CC BY-NC-SA 4.0}
\fancyfoot[L]{VS-Becke-V0.9-6}


\renewcommand{\headrulewidth}{0.4pt} % Fügt eine Linie unter der Kopfzeile hinzu
\renewcommand{\footrulewidth}{0.4pt} % Fügt eine Linie über der Fußzeile hinzu
\renewcommand{\headrulewidth}{0pt}   % Entfernt die Linie unter der Kopfzeile
\renewcommand{\footrulewidth}{0.4pt} % Fügt eine Linie über der Fußzeile hinzu


\definecolor{dkgreen} {rgb}{0,0.6,0}
\definecolor{gray}{rgb}{0.5,0.5,0.5}
\definecolor{mauve}{rgb}{0.58,0,0.82}

\lstset{frame=tb,
  language=Java,
  aboveskip=3mm,
  belowskip=3mm,
  showstringspaces=false,
  columns=flexible,
  basicstyle={\small\ttfamily},
  numbers=none,
  numberstyle=\tiny\color{gray},
  keywordstyle=\color{blue},
  commentstyle=\color{dkgreen},
  stringstyle=\color{mauve},
  breaklines=true,
  breakatwhitespace=true,
  tabsize=3
}

\usepackage[backend=biber,style=ieee]{biblatex}
\addbibresource{references.bib}
\providecommand{\main}{.}  % *Modification: define file location


\usepackage{subfiles}  % Best loaded last in the preamble

% Titel und Autor
\title{Verteilte Systeme\\ (Skript zum Modul v0.9-6)}
\author{Prof. Dr. Martin Becke}

\pgfplotsset{compat=1.18} 

\makeindex

%%%%%%%%%%%%%% Start Documents %%%%%%%%%%%%%%%%%%%%%%%
\begin{document}

% Titelseite
\maketitle
\vspace*{\fill} % Fügt vertikalen Raum hinzu, um das Bild in der Mitte zu zentrieren
\begin{figure}[h]
  \centering
  \includegraphics[width=0.25\textwidth]{fig/graphics/CADS_Logo_Quadrat_300x300_RGB_72dpi.jpg} % Passt die Breite des Bildes an, hier ist sie auf die Hälfte der Textbreite eingestellt.
  %\caption{Ihr Bildtitel}
\end{figure}
\vspace*{\fill}

\newpage

%%%%%%%%%%%%%% Copyright %%%%%%%%%%%%%%%%%%%%%%%
\noindent
\textcopyright\ 2022-2025, Prof. Dr. Martin Becke\footnote{Dieser Text wurde in Teilen mit der Unterstützung Sublime, Latex und Werkzeugen wie Rechtschreibprüfung, DeepL, ChatGPT (Sprache und Aufbau) und Claude (Code) erzeugt. Dieser Text{} wird durchgehend vom Autor geprüft, erweitert und verfeinert. Dieser Text ist nur für die Lehre vorgesehen. Bitte beachten Sie Lizenz und Version.}\\
Dieses Werk ist lizenziert unter der Creative Commons (4.0 International License)\\
- Namensnennung\\ - Nicht-kommerziell\\ - Weitergabe unter gleichen Bedingungen \\Um eine Kopie dieser Lizenz zu sehen, besuchen Sie\\ \url{http://creativecommons.org/licenses/by-nc-sa/4.0/}. 
Sonstig: Alle Rechte vorbehalten.
\noindent
\\\\
%%%%%%%%%%%%%% Version %%%%%%%%%%%%%%%%%%%%%%%
Versionen:\\
0.9-0 2023 SoSe - Initiale Version, noch stark Text-basiert. Schlichtes Layout\\
0.9-1 2023 SoSe - Weiter noch stark Text-basiert. Besseres Layout (DinA4, Header, Footer)\\
0.9-2 2023 SoSe - Erste Überarbeitung Text - Tag\\
0.9-3 2023 WiSe - Zweite überarbeitung Text - Tag \\
0.9-4 2023 SoSe - Dritte Überarbeitung Text - Format Beispiele\\
0.9-5 2025 SoSe - Vierte Überarbeitung Text - Angepasste Slides\\
0.9-6 2025 SoSe - Fünfte Überarbeitung Text - Format\\
%%%%%%%%%%%%%% Motivation %%%%%%%%%%%%%%%%%%%%%%%

Liebe Studierende,\\ vor uns liegt eine einzigartige Gelegenheit. Unser gemeinsames Skript, an dem bereits viele engagierte Geister gearbeitet haben, wartet darauf, durch unsere Ideen, Erkenntnisse und Beiträge weiter verbessert zu werden. Es ist ein lebendiges Werkzeug für das Lernen, und wir sind die aktuellen Hüter.

Sie haben alle die Möglichkeit, Ihren Stempel auf dieses Skript zu setzen und es noch besser zu machen. Durch Ihre Teilnahme an der Überarbeitung dieses Skripts tragen Sie dazu bei, die Qualität unserer Ausbildung zu verbessern und unser gemeinsames Wissen zu erweitern.

Indem Sie Ihre Ideen und Beobachtungen teilen, tragen Sie dazu bei, den Inhalt zu verfeinern und zu erweitern. Vielleicht haben Sie eine neue Methode oder ein neues Konzept entdeckt, das aufgenommen werden könnte. Oder vielleicht haben Sie eine Möglichkeit entdeckt, eine komplexe Idee klarer oder anschaulicher zu erklären. Jeder Beitrag, unabhängig von seiner Größe, wird dazu beitragen, dieses Skript zu einem noch besseren Lernwerkzeug zu machen.

Darüber hinaus bietet Ihnen die Mitarbeit an diesem Skript eine wertvolle Möglichkeit, Ihre Fähigkeiten in der Zusammenarbeit, der Wissenschaftskommunikation und der kritischen Analyse zu vertiefen. Sie werden durch die Diskussionen in der Lage sein, Ihr Wissen auf die Probe zu stellen und gleichzeitig neue Erkenntnisse zu gewinnen.

In diesem Prozess sind wir mehr als nur Einzelpersonen; wir sind Teil einer Gemeinschaft von Lernenden, die zusammenarbeiten, um ein gemeinsames Ziel zu erreichen. Der Erfolg dieses Skripts hängt von jedem von uns ab. Lassen Sie uns also unsere Kräfte bündeln, unsere Kreativität und unser Wissen nutzen, um dieses Skript zum Besten zu machen, was es sein kann.Auch wenn der Single-Threaded-Prozess von Node.js in vielen Situationen Vorteile bietet, hat er auch seine Grenzen. Zum Beispiel kann der Single-Threaded-Prozess die Rechenleistung von Mehrkernprozessoren ohne weiteres nicht voll ausnutzen, und rechenintensive Aufgaben können die Leistungsfähigkeit des Servers beeinträchtigen. In solchen Fällen kann es sinnvoll sein, Techniken wie Clustering, Worker-Threads oder alternative Architekturen zu verwenden, um die Leistung und Skalierbarkeit der Anwendung zu verbessern.

Ihr Beitrag wird nicht nur Ihnen und Ihren Kommilitonen zugutekommen, sondern auch zukünftigen Studierenden, die auf diesem Skript aufbauen werden. Lassen Sie uns dieses Erbe fortsetzen und gemeinsam ein Skript schaffen, auf das wir stolz sein können.
\\\\
Martin Becke
\newpage
\paragraph{Hall of Fame}
\mbox{}\\
%%%%%%%%%%%%%% Acknowledgment %%%%%%%%%%%%%%%%%%%%%%%
Dank für die Kommentare, Anmerkungen und Korrekturen:  \\
2023 SoSe Frank Matthiesen (Korrekturen Vortragsfolien) \\
2023 SoSe Niklas Kück (Viele Korrekturen und Textvorschläge für Kapitel 1,2 \& 3)\\
2023 SoSe Matz Heitmüller (Korrekturen: Beispiel-Architekturen und Folien Kapitel 2, Verschiedene Korrekturen) \\
2023 SoSe Adel Ahmad (Hinweis Pull Argumentation, fehlende Motivation IDM, Unschärfe NOS) \\
2023 SoSe Haidar Mortada (Fehler  Erasure-Coding) \\
2023 WiSe Morten Mednis (Fehler in Speedup Funktion) \\
2023 WiSe Yannis Günther (Rechtschreibung 26x) \\
2023 WiSe Sebastian Wewer (Rechtschreibung 38x, Position Graphik) \\
2023 WiSe Malte Behrmann (Rechtschreibfehler 2x)\\
2024 SoSe Luca Schmidt (Optimierung von verschiedenen Textstellen) \\
2024 SoSe Johannes Öhlers (Fehler: Openness, Event Driven)\\
2025 WiSe Jana Geske (Rechtschreibung 1x)\\
2025 WiSe Luca Stöver (Rechtschreibung 1x)\\
2025 WiSe Timon Buck (Optimierung Text Skalierung)\\
\\
%%%%%%%%%%%%%% Inhaltsverzeichnis %%%%%%%%%%%%%%%%%%%%%%%
\tableofcontents
\newpage
%%%%%%%%%%%%%%       Orga         %%%%%%%%%%%%%%%%%%%%%%%
\input{orga/vs-script-orga}
\newpage
% Einleitung
% Termin 1 - vs-slides-chapter01-1
%%%%%%%%%%%%%%       Chapter I      %%%%%%%%%%%%%%%%%%%%%%%
%%%%%%%%%%%%%%      Einleitung      %%%%%%%%%%%%%%%%%%%%%%%
\subfile{kapitel-01/chapter-01}
\newpage
%%%%%%%%%%%%%%      Chapter II      %%%%%%%%%%%%%%%%%%%%%%%
%%%%%%%%%%%%%%      Architektur     %%%%%%%%%%%%%%%%%%%%%%%
\subfile{kapitel-02/chapter-02}
\newpage

%%%%%%%%%%%%%%      Chapter III     %%%%%%%%%%%%%%%%%%%%%%%
%%%%%%%%%%%%%%      Realisierung    %%%%%%%%%%%%%%%%%%%%%%%
\subfile{kapitel-03/chapter-03}
\newpage

%%%%%%%%%%%%%%      Chapter IV      %%%%%%%%%%%%%%%%%%%%%%%
%%%%%%%%%%%%%%      Koordination    %%%%%%%%%%%%%%%%%%%%%%%
\subfile{kapitel-04/chapter-04}
\newpage

%%%%%%%%%%%%%%      Chapter IV      %%%%%%%%%%%%%%%%%%%%%%%
%%%%%%%%%%%%%%      Algorithmen     %%%%%%%%%%%%%%%%%%%%%%%
\subfile{kapitel-05/chapter-05}
\newpage

%%%%%%%%%%%%%%      Chapter IV      %%%%%%%%%%%%%%%%%%%%%%%
%%%%%%%%%%%%%%      Deployment      %%%%%%%%%%%%%%%%%%%%%%%
\subfile{kapitel-06/chapter-06}


%%%%%%%%%%%%%% Do not touch   %%%%%%%%%%%%%%%%%%%%%%%
%%%%%%%%%%%%%% close document %%%%%%%%%%%%%%%%%%%%%%%
\renewcommand{\indexname}{Stichwortverzeichnis}
\addcontentsline{toc}{section}{Stichwortverzeichnis}
\printindex
\newpage
% Literaturverzeichnis
\printbibliography

\end{document}

