\documentclass{beamer}

\usetheme{CambridgeUS}
\usecolortheme{dolphin}

\title{Verteilte Systeme}
\subtitle{Konsens, Fehlertoleranz und Replikation}
\author{Prof. Dr. Martin Becke}
\date{\today}

\begin{document}

\begin{frame}
    \titlepage
\end{frame}

\begin{frame}{Konsensbildung in verteilten Systemen}
    Konsensbildung ist der Prozess, bei dem eine Gruppe von Knoten in einem verteilten System eine gemeinsame Vereinbarung über einen Wert oder Zustand trifft. \newline
    \textbf{Wesentliche Herausforderungen:}
    \begin{itemize}
        \item Fehlertoleranz.
        \item Kommunikationslatenz.
        \item Sicherheit.
    \end{itemize}
    \textbf{Analogie:} Wie eine Jury, die ein Urteil fällen muss – alle Geschworenen müssen sich einigen, auch wenn einige anderer Meinung sind oder nicht erreichbar sind.
\end{frame}

\begin{frame}{Quorumsabstimmung}
    Bei der Quorumsabstimmung muss eine Mindestanzahl von Knoten (das Quorum) zustimmen, um eine Entscheidung zu treffen. \newline
    \textbf{Kompromiss:}
    \begin{itemize}
        \item Kleineres Quorum: Schnellere Entscheidungen, aber höhere Fehlerwahrscheinlichkeit.
        \item Größeres Quorum: Höhere Zuverlässigkeit, aber langsamere Entscheidungen.
    \end{itemize}
\end{frame}

\begin{frame}{Zentralisierte vs. dezentrale Konsensbildung}
    \begin{itemize}
        \item \textbf{Zentralisiert:} Ein Knoten (Leader) steuert den Prozess. Einfach, aber anfällig für Ausfälle des Leaders. \newline \textit{Analogie:} Wie eine Monarchie.
        \item \textbf{Dezentralisiert:} Alle Knoten sind gleichberechtigt. Robuster und skalierbarer, aber komplexer. \newline \textit{Analogie:} Wie eine Demokratie.
    \end{itemize}
    Hybride Ansätze kombinieren beide Strategien.
\end{frame}

\begin{frame}{Fallbeispiel Zentral: ZooKeeper Atomic Broadcast (ZAB)}
    ZAB ist ein Konsensprotokoll, das von Apache ZooKeeper verwendet wird. \newline
    \textbf{Eigenschaften:}
    \begin{itemize}
        \item Garantiert zuverlässige Replikation von Updates.
        \item Ermöglicht geordnete Vereinbarungen über Änderungen am Systemzustand.
    \end{itemize}
    \textbf{Analogie:} Wie ein Orchesterdirigent, der sicherstellt, dass alle Musiker im gleichen Takt spielen.
\end{frame}

\begin{frame}{Fallbeispiel Dezentral: Blockchain-Mechanismen}
    Blockchain-Mechanismen ermöglichen dezentrale Konsensbildung. \newline
    \textbf{Beispiele:}
    \begin{itemize}
        \item \textbf{Proof-of-Work (PoW):} Miner lösen komplexe Rätsel, um Transaktionen zu validieren und neue Blöcke hinzuzufügen.
        \item \textbf{Proof-of-Stake (PoS):} Die Wahrscheinlichkeit, einen Block hinzuzufügen, hängt vom Anteil am Netzwerk ab.
    \end{itemize}
\end{frame}

\begin{frame}{Blockchain: Vorteile und Nachteile}
    \textbf{Vorteile:}
    \begin{itemize}
        \item Dezentralisierung.
        \item Transparenz.
        \item Sicherheit.
    \end{itemize}
    \textbf{Nachteile:}
    \begin{itemize}
        \item Skalierbarkeit.
        \item Hoher Energieverbrauch (bei PoW).
        \item Anfälligkeit für 51%-Angriffe.
        \item Datenunveränderlichkeit (Problematisch bei Fehlern).
    \end{itemize}
\end{frame}

\begin{frame}{Fehlertoleranz}
    Fehlertoleranz ist die Fähigkeit eines Systems, trotz Fehlern funktionsfähig zu bleiben. \newline
    \textbf{Strategien:}
    \begin{itemize}
        \item Redundanz.
        \item Wiederherstellung.
        \item Fehlertolerante Kommunikation.
        \item Byzantinische Fehlertoleranz.
    \end{itemize}
\end{frame}

\begin{frame}{Fehlerarten}
    Fehler in verteilten Systemen können auftreten als:
    \begin{itemize}
        \item \textbf{Hardwarefehler:} Ausfall von Komponenten.
        \item \textbf{Softwarefehler:} Bugs im Code.
        \item \textbf{Kommunikationsfehler:} Verlust oder Verzögerung von Nachrichten.
        \item \textbf{Byzantinische Fehler:} Fehlerhafte oder böswillige Knoten.
    \end{itemize}
\end{frame}

\begin{frame}{Fehlermodell-Terminologie}
    Unterschiedliche Begriffe beschreiben verschiedene Aspekte von Fehlern:
    \begin{itemize}
        \item \textbf{Fault (Fehler):} Ursache des Problems.
        \item \textbf{Error (Fehlzustand):} Inkonsistenter Systemzustand.
        \item \textbf{Failure (Ausfall):} Beobachtbares Ergebnis.
    \end{itemize}
    Diese Begriffe sind wichtig für die Analyse und Behebung von Fehlern.
\end{frame}

\begin{frame}{Fehlerarten: Fail-Stop, Fail-Noisy, Fail-Silent}
    \begin{itemize}
        \item \textbf{Fail-Stop:} Der Knoten hält an und signalisiert den Fehler.
        \item \textbf{Fail-Noisy:} Der Knoten liefert falsche Ergebnisse.
        \item \textbf{Fail-Silent:} Der Knoten hält an, ohne den Fehler zu signalisieren.
    \end{itemize}
\end{frame}

\begin{frame}{Redundanz (1/2)}
    Redundanz erhöht die Fehlertoleranz durch zusätzliche Ressourcen:
    \begin{itemize}
        \item \textbf{Datenredundanz:} Mehrere Kopien der Daten (Replikation, Erasure Coding).
        \item \textbf{Rechenredundanz:} Parallele Ausführung von Berechnungen.
    \end{itemize}
\end{frame}

\begin{frame}{Redundanz (2/2)}
    \begin{itemize}
        \item \textbf{Kommunikationsredundanz:} Mehrere Kommunikationspfade.
        \item \textbf{Prozess-Resilienz:} System bleibt trotz Prozessausfällen funktionsfähig.
    \end{itemize}
\end{frame}

\begin{frame}{Replikation}
    Replikation erstellt und verwaltet Kopien von Daten auf mehreren Knoten. \newline
    \textbf{Vorteile:}
    \begin{itemize}
        \item Erhöhte Verfügbarkeit.
        \item Bessere Zuverlässigkeit.
        \item Verbesserte Skalierbarkeit.
        \item Schnellere Wiederherstellung.
    \end{itemize}
\end{frame}

\begin{frame}{Replikation: Strategien}
    \begin{itemize}
        \item \textbf{Passive Replikation (Master-Slave):} Ein Knoten (Master) bearbeitet Anfragen, andere Knoten (Slaves) werden synchronisiert.
        \item \textbf{Aktive Replikation (Multi-Master):} Alle Knoten bearbeiten Anfragen, erfordert komplexe Konsistenzverwaltung.
    \end{itemize}
\end{frame}

\begin{frame}{Replikation: Cold, Warm, Hot}
    \textbf{Unterschiede in der Aktualität der Replikate:}
    \begin{itemize}
        \item \textbf{Cold:} Seltene Aktualisierung (z. B. Backups).
        \item \textbf{Warm:} Regelmäßige Aktualisierung, aber nicht in Echtzeit.
        \item \textbf{Hot:} Nahezu Echtzeit-Aktualisierung, hohe Verfügbarkeit.
    \end{itemize}
\end{frame}

\begin{frame}{Erasure Coding}
    Erasure Coding teilt Daten in Fragmente und erstellt Paritätsinformationen. \newline
    \textbf{Vorteile:}
    \begin{itemize}
        \item Speicherplatzsparender als Replikation.
        \item Ermöglicht Datenwiederherstellung bei Teilverlust.
    \end{itemize}
    \textbf{Analogie:} Wie ein RAID-System, bei dem Daten über mehrere Festplatten verteilt werden.
\end{frame}

\end{document}