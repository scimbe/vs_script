\section{Veranstaltung VS}
\subsection{Anforderungen}
\begin{frame}
  \frametitle{Anforderungen für Einstieg VS}
  \framesubtitle{Was sollte ich mitbringen? }
  \begin{itemize}
    \item Interesse daran, wie verteilte Anwendungen funktionieren
    \item Ein wenig Englisch ist hilfreich
    \item Ausreichende Kenntnisse der Programmiersprachen Java oder C++
    \item Kenntnisse der Module: AD, DB, Programmier*, SE, BS, RN 
  \end{itemize}
\end{frame}
\subsection{Resourcen}
\begin{frame}
  \frametitle{Resourcen}
  \framesubtitle{Wo finde ich was?}
  \begin{itemize}
    \item Zentraler Punkt: MS Teams (Code: j9sy753)
    \item Zentrales Script: NEW Version 1.0.1 (Stabiler Inhalt, Format und Sprache werden weiter optimiert)
    \item Referenzliteratur und Basis früherer Vorlesungen \url{https://www.distributed-systems.net/index.php/books/ds4/}
    \item Folien mit der Struktur der Vorlesung 
  \end{itemize}
\end{frame}

\subsection{Prüfung}
\begin{frame}
  \frametitle{Prüfungsform}
  \framesubtitle{Abhängig der Teilnehmer}
  \begin{itemize}
    \item Mündliche Prüfung
    \item Basis das Praktikum
  \end{itemize}
\end{frame}

\subsection{Praktikum}
\begin{frame}
  \frametitle{Praktikumsinformationen}
  \framesubtitle{Entwicklungszyklus in VS - Empfehlung}
  \begin{itemize}
    \item Mit Standalone Applikation Design starten
    \item Trennen in Client/-Server mit Middleware
    \item Orchestrieren mit RPC-Architektur 
    \item Optimieren mit verteilten Algorithmus
  \end{itemize}
\end{frame}

\subsection{Praktikum - Best Practice}
\begin{frame}
  \frametitle{Praktikumshinweise}
  \framesubtitle{Was sollte ich beachten?}
  \begin{itemize}
    \item Maximale Gruppengröße 4 (auch 3 möglich/ 5 unter besonderen Umständen).
    \item Praktikum ist zeitaufwendig 
    \item Praktikum ist ein iterativer Prozess 
    \item Praktikum hat keine Anwesenheitspflicht, nur verpflichtende Protokolle
  \end{itemize}
\end{frame}

\subsection{PVL Kriterien}
\begin{frame}
  \frametitle{PVL}
  \framesubtitle{Anforderungen}
  \begin{itemize}
    \item Vollständige Dokumentation (Vorschlag ARC42) - Dokumentation muss zum Code passen und Code zur Dokumentation
    \item Dokumentation sollte midestens Komponenten, Klassen und Interaktionen umfassen (Vorschlag UML 2.5)
    \item Code Base, mit Test (Martin Becke und Frank Matthiesen müssen eingetragene Developer sein)
    \item Funktionale Umsetzung der beschriebenen Use Cases 
    \item Es sollten alle VS Standards nach Tanenbaum/van Steen eingehalten werden, oder wie sie auch im Skript beschrieben sind
    \item Layer und Tiers
  \end{itemize}

\end{frame}