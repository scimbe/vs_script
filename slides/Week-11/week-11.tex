\documentclass{beamer}

\usetheme{CambridgeUS}
\usecolortheme{dolphin}

\title{Verteilte Systeme}
\subtitle{Algorithmen für Koordination und Konsens}
\author{Prof. Dr. Martin Becke}
\date{\today}

\begin{document}

\begin{frame}
    \titlepage
\end{frame}

\begin{frame}{Algorithmen in Verteilten Systemen}
    Verteilte Systeme benötigen spezielle Algorithmen, um Herausforderungen wie Konsens, Koordination, Replikation und Fehlertoleranz zu meistern. \newline
    \textbf{Herausforderungen:}
    \begin{itemize}
        \item Knoten können ausfallen.
        \item Nachrichten können verloren gehen.
        \item Kommunikation ist oft asynchron.
    \end{itemize}
    \textbf{Analogie:} Wie ein Team von Bergsteigern: Sie müssen zusammenarbeiten und kommunizieren, um das Ziel zu erreichen, selbst wenn einige Mitglieder ausfallen oder der Weg unvorhersehbar wird.
\end{frame}

\begin{frame}{Paralleles Rechnen}
    Paralleles Rechnen ermöglicht die gleichzeitige Ausführung von Berechnungen auf mehreren Prozessoren oder Knoten. \newline
    \textbf{Eigenschaften:}
    \begin{itemize}
        \item Ideal für rechenintensive Aufgaben, die in unabhängige Teile zerlegt werden können.
        \item Beispiele: Matrixmultiplikation, Bildverarbeitung, Simulationen, Maschinelles Lernen.
    \end{itemize}
    \textbf{Analogie:} Wie ein Team von Köchen, die gemeinsam ein Menü zubereiten: Jeder Koch arbeitet an einem anderen Gang, wodurch das Menü schneller fertig wird.
\end{frame}

\begin{frame}{MapReduce}
    MapReduce ist ein Programmiermodell für die Verarbeitung großer Datenmengen in verteilten Systemen. \newline
    \textbf{Phasen:}
    \begin{itemize}
        \item \textbf{Map:} Die Eingabedaten werden in Schlüssel-Wert-Paare umgewandelt. \newline \textit{Analogie:} Sortieren von Wäsche nach Farben.
        \item \textbf{Reduce:} Paare mit gleichem Schlüssel werden zusammengefasst. \newline \textit{Analogie:} Zählen der Socken jeder Farbe.
    \end{itemize}
    \textbf{Beispiel:} Wortzählung in einem großen Textkorpus.
\end{frame}

\begin{frame}{Resilient Distributed Datasets (RDDs) - Spark}
    RDDs sind unveränderliche, verteilte Datensammlungen in Apache Spark, die effiziente Datenverarbeitung und Wiederverwendung ermöglichen. \newline
    \textbf{Funktionen:}
    \begin{itemize}
        \item Unterstützung für Transformationen wie \texttt{map}, \texttt{filter} und \texttt{reduce}.
        \item Geeignet für komplexe Operationen wie SQL-Abfragen oder Machine Learning.
    \end{itemize}
    \textbf{Analogie:} Wie ein gut organisiertes Lager, in dem Materialien (Daten) effizient abgerufen und verarbeitet werden können.
\end{frame}

\begin{frame}{Zentralisierte vs. Dezentralisierte Algorithmen}
    \begin{itemize}
        \item \textbf{Zentralisiert:} Ein einzelner Knoten steuert den Algorithmus. \newline \textit{Vorteil:} Einfachheit. \newline \textit{Nachteil:} Anfällig für Ausfälle des zentralen Knotens. \newline \textit{Analogie:} Ein Dirigent leitet ein Orchester.
        \item \textbf{Dezentralisiert:} Verantwortung verteilt auf alle Knoten. \newline \textit{Vorteil:} Robuster und skalierbarer. \newline \textit{Nachteil:} Komplexer in der Umsetzung. \newline \textit{Analogie:} Ein Jazz-Ensemble improvisiert zusammen.
    \end{itemize}
\end{frame}

\begin{frame}{Konsensalgorithmen}
    Konsensalgorithmen stellen sicher, dass alle Knoten in einem verteilten System trotz Fehlern denselben Zustand oder Wert annehmen. \newline
    \textbf{Beispiele:}
    \begin{itemize}
        \item Paxos.
        \item Raft.
        \item ZAB (ZooKeeper Atomic Broadcast).
    \end{itemize}
\end{frame}

\begin{frame}{Paxos}
    Paxos ist ein bekannter Konsensalgorithmus, der robust und fehlertolerant ist, jedoch als komplex in der Implementierung gilt. \newline
    \textbf{Rollen:}
    \begin{itemize}
        \item Proposer (Antragsteller).
        \item Acceptor (Annehmer).
        \item Learner (Lerner).
    \end{itemize}
    \textbf{Phasen:} Vorbereitung, Annahme, Lernen.
\end{frame}

\begin{frame}{Raft}
    Raft ist eine modernere und verständlichere Alternative zu Paxos. \newline
    \textbf{Kernkonzepte:}
    \begin{itemize}
        \item \textbf{Leader Election:} Ein Leader wird gewählt.
        \item \textbf{Log Replication:} Der Leader repliziert Logs an andere Knoten.
        \item \textbf{Safety:} Garantiert Konsistenz.
    \end{itemize}
    \textbf{Analogie:} Wie ein gut organisiertes Team, das von einem gewählten Anführer koordiniert wird.
\end{frame}

\begin{frame}{ZooKeeper Atomic Broadcast (ZAB)}
    ZAB ist ein speziell für Apache ZooKeeper entwickeltes Konsensprotokoll. \newline
    \textbf{Eigenschaften:}
    \begin{itemize}
        \item Garantiert zuverlässige Replikation.
        \item Sichert konsistenten Systemzustand.
    \end{itemize}
    \textbf{Modi:}
    \begin{itemize}
        \item Recovery (Wiederherstellung).
        \item Broadcast (Übertragung).
    \end{itemize}
    \textbf{Analogie:} Wie ein Nachrichtendienst, der sicherstellt, dass alle Abonnenten dieselben Nachrichten erhalten.
\end{frame}

\begin{frame}{Koordination: Snapshots und Checkpoints}
    \begin{itemize}
        \item \textbf{Snapshots:} Erfassen den globalen Systemzustand zu einem bestimmten Zeitpunkt. \newline \textit{Analogie:} Wie ein Foto.
        \item \textbf{Checkpoints:} Speichern den lokalen Prozesszustand. \newline \textit{Analogie:} Wie ein Speicherpunkt in einem Videospiel.
    \end{itemize}
    Beide Techniken sind essenziell für Fehlertoleranz und Debugging.
\end{frame}

\begin{frame}{Koordination: Deadlock Detection und Termination Detection}
    \begin{itemize}
        \item \textbf{Deadlock Detection:} Erkennt Situationen, in denen Prozesse zyklisch aufeinander warten.
        \item \textbf{Termination Detection:} Erkennt, wann alle Prozesse ihre Arbeit abgeschlossen haben.
    \end{itemize}
    Beide Mechanismen sind entscheidend für die korrekte Funktion verteilter Systeme.
\end{frame}

\begin{frame}{Koordination: Mutual Exclusion und Barriers}
    \begin{itemize}
        \item \textbf{Mutual Exclusion:} Verhindert, dass mehrere Prozesse gleichzeitig auf eine Ressource zugreifen. \newline \textit{Analogie:} Wie ein Schlüssel für ein Badezimmer.
        \item \textbf{Barriers:} Synchronisiert Prozesse an bestimmten Punkten im Code. \newline \textit{Analogie:} Wie eine Startlinie bei einem Rennen.
    \end{itemize}
\end{frame}

\begin{frame}{Koordination: Self-Stabilization und Smart Contracts}
    \begin{itemize}
        \item \textbf{Self-Stabilization:} Systeme erholen sich automatisch von Fehlern und erreichen einen korrekten Zustand. \newline \textit{Analogie:} Wie ein Stehaufmännchen.
        \item \textbf{Smart Contracts:} Automatisierte Verträge auf Blockchains. \newline \textit{Analogie:} Wie ein digitaler Notar.
    \end{itemize}
\end{frame}

\end{document}