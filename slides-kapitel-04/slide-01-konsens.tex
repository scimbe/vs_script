section\section{Koordination: Konsens, Synchronisation und Fehler}
\subsection{Koordination}
\begin{frame}
  \frametitle{Koordination}
  \framesubtitle{Prozess mit verschiedenen Zielen}
  \begin{itemize}
    \item Kooperation
    \item Informationsaustausch
    \item Konsistentes Verhalten 
    \item Kohärentes Verhalten
  \end{itemize}
\end{frame}

\begin{frame}
  \frametitle{Koordination}
  \framesubtitle{Anforderungen}
  \begin{itemize}
    \item Einigung (Konsensbildung)
    \item Synchronisation (Globale Ordnung)
    \item Fehlerhandling und Recovery
  \end{itemize}
\end{frame}

\subsection{Konsens}
\begin{frame}
  \frametitle{Konsens}
  \framesubtitle{Idee}
  \begin{itemize}
    \item Gleiche Sicht auf die Daten
    \item Selbst Licht zu langsam (Rechnung im Script)
    \item Nicht endgültig lösbar
  \end{itemize}
\end{frame}


\begin{frame}
  \frametitle{CAP Theorem}
  \framesubtitle{Idee}
  \begin{itemize}
    \item Konsistenz (C)
    \item Verfügbarkeit (A)
    \item Partitionstoleranz (P)
    \item Herausforderung: C vs A
  \end{itemize}
\end{frame}

\begin{frame}
  \frametitle{Konsensmodelle}
  \framesubtitle{Ideen Umsetzung}
  \begin{itemize}
    \item Tunable Consistency
    \item Quorum-Systeme
    \item CRDTs 
    \item Hybride Ansätze
  \end{itemize}
\end{frame}

\begin{frame}
  \frametitle{Konsensmodelle}
  \framesubtitle{Ideen Definition}
  \begin{itemize}
    \item Keine feste Definition, eher eine Orientierung
    \item Datenzentrische Konsistenzmodelle
    \item Client-zentrische Konsistenzmodelle
    \item Alternative Konsistenzmodelle
  \end{itemize}
\end{frame}


\begin{frame}
  \frametitle{Konsensmodelle}
  \framesubtitle{Datenzentrische Konsistenzmodelle}
  \begin{itemize}
    \item Strong Consistency (Traum)
    \item Linearizability Consistency (Realisitisch?)
    \item Sequential Consistency 
    \item Causal Consistency
    \item Eventual Consistency
  \end{itemize}
\end{frame}


\begin{frame}
  \frametitle{Datenzentrische Konsistenzmodelle}
  \framesubtitle{Atomare Konsistenz}
  \begin{itemize}
    \item Gleich Linearizability Consistency, manchmal als Strong Consistency interpretiert
    \item Protokolle:  Paxos, Raft oder Zab
    \item Grundidee: Definition eines Zyklus 
    \item Beispiel der Anwendung ist Finanztransaktionssystemen
  \end{itemize}
\end{frame}

\begin{frame}
  \frametitle{Datenzentrische Konsistenzmodelle}
  \framesubtitle{Sequentielle Konsistenz}
  \begin{itemize}
    \item Operationen in der gleichen Reihenfolge 
    \item Synchronisation der Knoten kann verzögert erfolgen
    \item Ausgewogene Mischung aus Konsistenz und Leistung  
    \item Keine Kausalität
  \end{itemize}
\end{frame}


\begin{frame}
  \frametitle{Datenzentrische Konsistenzmodelle}
  \framesubtitle{Kausale Konsistenz}
  \begin{itemize}
    \item Kausalen Beziehungen zwischen Operationen
    \item Nur kausale Beziehung bei Operationen mit Abhängigkeit
    \item Beispiel Tweet mit Kommentaren und alternativer Tweet 
    \item Kausaler Konsistenz basiert in Implementierung gerne auf logische Uhren
  \end{itemize}
\end{frame}

\begin{frame}
  \frametitle{Datenzentrische Konsistenzmodelle}
  \framesubtitle{Eventual Consistency}
  \begin{itemize}
    \item Daten vorübergehend inkonsistent (Nicht als eventuell übersetzen)
    \item Fokus auf höherer Verfügbarkeit (C)
    \item Beispielhaft umgesetzt in DynamoDB
    \item Immer höhere Verbreitung
  \end{itemize}
\end{frame}


\begin{frame}
  \frametitle{Client-zentrische Konsistenzmodelle}
  \framesubtitle{Monotonic Reads}
  \begin{itemize}
    \item Liest niemals ältere Daten, die er zuvor gelesen hat 
    \item Liest gleiche oder neuere Daten unabhängig seiner Position
    \item Beispielhaft Zeitstempel oder Versionsnummer bei Read
  \end{itemize}
\end{frame}


\begin{frame}
  \frametitle{Client-zentrische Konsistenzmodelle}
  \framesubtitle{Monotonic Writes}
  \begin{itemize}
    \item Schreibvorgänge in der Reihenfolge des Clients
    \item Beispielhaft: Zeitstempeln oder Versionsnummern bei Write 
    \item Beispiel Banktransaktionen
  \end{itemize}
\end{frame}

\begin{frame}
  \frametitle{Client-zentrische Konsistenzmodelle}
  \framesubtitle{Read your Writes}
  \begin{itemize}
    \item Benutzer sieht immer seine Änderungen
    \item Beispielhaft eitstempeln oder Versionsnummern bei Wrtie 
    \item Beispiel Statusaktualisierungen bei sozialen Netzen
  \end{itemize}
\end{frame}


\begin{frame}
  \frametitle{Client-zentrische Konsistenzmodelle}
  \framesubtitle{Writes follow Reads}
  \begin{itemize}
    \item Kausale Konsistenz und das „Writes follow Reads“-Prinzip sind eng verbunden
    \item  Schreiboperationen, die auf Leseoperationen basieren, sind in logischer konsistenter Reihenfolge
    \item Beispiel ist Cassandra 
    \item Google Docs verwendet das Operational Transformation (OT) Framework 
  \end{itemize}
\end{frame}


\begin{frame}
  \frametitle{Konsistensmodell}
  \framesubtitle{Alternativen}
  \begin{itemize}
    \item Abweichung ACID-Eigenschaften (Atomicity, Consistency, Isolation, Durability)
    \item BASE (Basically Available, Soft state, Eventually consistent)
    \item PACELC (Partition-tolerance, Availability, Consistency, Eventual consistency, Latency, and Consistency trade-offs) 
  \end{itemize}
\end{frame}

\begin{frame}
  \frametitle{Konsistensmodell}
  \framesubtitle{Alternativen}
  \begin{itemize}
    \item Continuous Consistency 
    \begin{itemize}
      \item CONIT-Modell
    \end{itemize}
    \item Fork Consistency
    \item Multidimensional Consistency
    \item Adaptable Consistency
    \item View Consistency
  \end{itemize}
\end{frame}

\begin{frame}
  \frametitle{Konsistenssichten}
  \framesubtitle{Write Write Conflicts}
  \begin{itemize}
    \item Option 1: Sperren und Transaktionen
    \item Option 2: Optimistische Nebenläufigkeitskontrolle
    \item Option 3: Versionskontrolle und Zusammenführungsstrategien
  \end{itemize}
\end{frame}