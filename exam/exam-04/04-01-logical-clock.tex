\documentclass{article}
\usepackage{enumitem}

\begin{document}

\title{Aufgabe: Logische Uhren und Inter-Server-Kommunikation}
\maketitle

Gegeben sind vier Server: Server A, B und C, sowie ein Monitor-Server M. Die Server A, B und C kommunizieren untereinander und senden dabei asynchron Nachrichten. Der Monitor-Server M protokolliert das Senden und Empfangen dieser Nachrichten. Jede gesendete und empfangene Nachricht wird durch eine logische Uhr gekennzeichnet, die den Zeitstempel der jeweiligen Aktion darstellt.
\\\\
Die Protokolleintragungen beinhalten die folgenden Informationen:
\begin{enumerate}
  \item Name des Senderservers (A, B oder C)
  \item Name des Empfängerservers (A, B oder C)
  \item Zeitstempel der logischen Uhr zum Zeitpunkt des Sendens
  \item Zeitstempel der logischen Uhr zum Zeitpunkt des Empfangens
  \item Inhalt der Nachricht
\end{enumerate}
Für die Kommunikation und Protokollierung gelten folgende Regeln:
\begin{enumerate}
  \item Jeder Server kann zu beliebigen Zeitpunkten Nachrichten senden und empfangen.
  \item Die Eintragungen im Protokoll können in beliebiger Reihenfolge am Monitor-Server M ankommen.
  \item Jede Aktion (Senden oder Empfangen einer Nachricht), die von einem Server ausgeführt wird, erhöht die logische Uhr dieses Servers um eins.
  \item Wenn eine Nachricht von einem Server an einen anderen Server gesendet wird, wird der aktuelle Wert der logischen Uhr dieses Servers in das Protokoll für das Senden eingefügt.
  \item Wenn eine Nachricht von einem Server empfangen wird, wird der aktuelle Wert der logischen Uhr dieses Servers in das Protokoll für das Empfangen eingefügt.
\end{enumerate}
Betrachten Sie das folgende Beispiel für eine Logdatei:
\begin{enumerate}
  \item A sendet an B, Senden: [A:1], Empfangen: [B:2]
  \item A sendet an C, Senden: [A:2], Empfangen: [C:3]
  \item B sendet an A, Senden: [B:3], Empfangen: [A:4]
  \item C sendet an A, Senden: [C:4], Empfangen: [A:5]
  \item C sendet an B, Senden: [C:5], Empfangen: [B:6]
  \item B sendet an C, Senden: [B:7], Empfangen: [C:8]
  \item A sendet an B, Senden: [A:6], Empfangen: [B:8]
  \item B sendet an A, Senden: [B:9], Empfangen: [A:10]
  \item C sendet an A, Senden: [C:9], Empfangen: [A:11]
\end{enumerate}
Ihre Aufgabe besteht darin:
\begin{enumerate}
  \item Rekonstruieren Sie die logischen Uhren der einzelnen Server basierend auf den angegebenen Zeitstempeln in der Logdatei.
  \item Entwickeln Sie eine alternative, gültige Permutation der Reihenfolge der Eintragungen in der Logdatei.
  \item Begründen Sie, warum diese alternative Permutation gültig ist unter Berücksichtigung der Eigenschaften von logischen Uhren.
\end{enumerate}
Dies sollte Ihnen dabei helfen, ein tieferes Verständnis dafür zu bekommen, wie logische Uhren in einem verteilten System verwendet werden können, um die relative Ordnung von Ereignissen zu bestimmen, insbesondere wenn es um die Kommunikation zwischen verschiedenen Servern geht. Es sollte Ihnen auch dabei helfen zu verstehen, wie verschiedene Reihenfolgen der Ereignisse die logischen Uhren der einzelnen Server beeinflussen können.

\end{document}